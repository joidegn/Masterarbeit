\documentclass[11pt]{article}
%Gummi|063|=)
\usepackage{amsmath}
\usepackage[utf8]{inputenc}
\usepackage[authoryear, round]{natbib}
\usepackage{graphicx}

\title{\textbf{Predicting Economic Time-Series using Dynamic Factor Models under Structural Breaks}}
\author{Johannes Degn}
\date{01.08.2013}
\begin{document}


\maketitle

\tableofcontents

\section{Introduction}

The idea behind factor models or diffusion index models is that there are underlying latent factors which explain the evolution of the observable variables $X$. 

One important advantage of factor models is that they do not necessarily suffer if the fraction $\frac{T}{N}$ becomes small whereas traditional models can usually not easily cope with more variables than observations.

Bootstrap the structural break tests!!! See if performance can be increased (i.e. improve on the power and maybe the size of the tests of Breitung and Eickmeier)

Further provide a horserace to see which of the following is better for forecasting:
\begin{enumerate}
	\item A static factor model with no structural breaks (but with more factors)
	\item A static factor model with structural break(s)
	\item A dynamic factor model with no structural breaks
	\item A dynamic factor model with structural break(s)
\end{enumerate}


\section{Factor Models}


Factor Models are models of the form 

\begin{equation}
	\label{factor equation}
	X = F \Lambda' + e = C + e
\end{equation}
Where $X$ is a $T \times N$ Matrix of predictors. $F$ is a $T \times r$ factor matrix. $\Lambda$ is a $N \times r$ loadings matrix. $e$ is a $T \times N$ matrix of idiosyncratic errors. $C$ is called the common component. (\ref{factor equation}) is also called factor equation.
Depending on whether authors index by row, column or both, several alternative ways of writing down the factor model equation are used in the literature.

\begin{equation}
	\label{factor equation, it indexed}
	X_{it} = \lambda_i' F_t + e_{it}
\end{equation}
\begin{equation}
	\label{factor equation, t indexed}
	X_t = \Lambda F_t + e_t
\end{equation}
\begin{equation}
	\label{factor equation, i indexed}
	X_i = F_i \lambda + e_i
\end{equation}


\begin{equation}
	\label{factor equation, lag polynomial}
	X_t = \lambda(L) f_t + e_t
\end{equation}
\begin{equation}
	f_t = \Psi(L) f_{t-1} + \eta_t
\end{equation}
Where $X_t$ is a $N \times 1$ vector of potentially very many predictors. $\lambda(L)$ is the $N \times q$ matrix of lag polynomials, $f_t$ is a $q \times 1$ vector of latent factors. $e_t$ is a $N \times 1$ vector of idiosyncratic errors which may or may not be serially auto-correlated.

There 


\subsection{Static Factor Models}
For static factor models the idiosyncratic component is a white noise process (i.e. independent)

\subsubsection{Estimation of the Number of Factors}
\subsection{Dynamic Factor Models}
Dynamic factor models relax the assumption that the indiosyncratic innovation terms are white noise (i.e. independent and mean zero). In most applications the error terms will follow some time dependence structure. Dynamic Factor Models are also called generalized factor models by . Dynamic Factor Models can be written as in equation \ref{var-representation of factors} where the factor loadings of the static factor model has been replaced by dynamic factor loadings (i.e. a polynomial of the lag operator). 

\begin{equation}
	\label{var-representation of factors}
	x_{it} = \lambda_i'(L) f_t
\end{equation}
Where $\lambda_i(L)$ is a dynamic matrix of loadings (i.e. $L$ is the lag operator).
\begin{equation}
	\label{dynamic factors equation}
	f_t = \Gamma f_{t-1} \text{\hspace{1cm} (VAR representation)}
\end{equation}

Dynamic factor models can be written as static factor models by augmenting the static factors

\citet{breitung2004identification} also call the $f_t$ structural factors which are gotten from the static or "reduced form" factors $F_t$.
\citet{giannone2002tracking} and \citet{forni2009opening} apply principal component analysis to the residual covariance matrix for equation (\ref{var-representation of factors}) [see page 4 of \citet{breitung2004identification}]. \citet{bai2007determining} propose a method to estimate the $q$ dynamic factors from the $r$ static factors. Their method involves first estimating a VAR of the static factors $\hat F_t$ on its lags. The order of this VAR will be equal to the $q$ static factors.

\citet{geweke1977dynamic} and \citet{sargent1977business} distinguish between exact DFMs and approximate DFMs. Approximated DFMs allow for correlatedness of the idiosyncratic innovations $e_{it}$ between periods. I.e. $E(e_{it}e_{js})$ is allowed to be different from $0$ but only in a limited way.


\subsubsection{Estimation of the Number of Lags}
\citep{breitung2004identification}

\citet{bai2007determining} propose to compare the residual variance of a regression of the factor estimates $\hat F_t$ on their lags to a shrinking bound.

\subsection{Forecasting}

\begin{equation}
	\label{forecasting equation}
	y_{t+h}^h = \mu + \alpha(L) y_t + b(L) \hat f_t + e_{t+h}^h
\end{equation}
So forecasts rely on lags of the variable and estimated of the common factors.

\section{Notes on the Literature}
\citet{cragg1997inferring} shows in a Monte Carlo analysis that $T\rightarrow\infty$ and $N\rightarrow\infty$ makes that classical theory for predicting the number of factors performs badly. Resultingly information criteria as in \citet{bai2002determining} should be used. These results hold under heteroskedasticity and weak serial correlation. But: structural breaks!

Information criteria in \citet{bai2002determining} are a generalized version of Mallows C$_p$ \citep{mallows1973some} (Panel C$_p$ or PC$_p$)

Principal Components estimation of the factors is asymptotically equivalent to maximum likelihood estimation if normality is assumed \citep{bai2003inferential}.


\appendix
\addcontentsline{toc}{section}{Appendix}

\section{Derrivation of Principal Components}

We want to find a vector $u$ such that the the variance along the projection of $x_i$ onto $u$ is maximized where $u$ is a unit norm vector.
$$\underset{u: ||u|| = 1}{\max} \ \frac{1}{T} \sum_{i=1}^T(x_i'u)^2 = \underset{u: u'u = 1}{\max} \ u' ( \frac{1}{T} \sum_{i=1}^T x_i'x_i )u = \underset{u: u'u = 1}{\max} \ u' (\Sigma)u$$
Setting up the Lagrangian yields
$$ L(u, \lambda) = u' \Sigma - \lambda(u'u-1)$$
After taking derrivatives with respect to $u$ we are left with
$$\frac{\partial L}{\partial u} = \Sigma u -\lambda u \overset{!}{=} 0$$ or in other words $u$ is an eigenvector of the covariance matrix $\Sigma$ with corresponding eigenvalue $\lambda$.


\bibliographystyle{plainnat}
\bibliography{/home/joi/workspace/latex/citations}
\end{document}
